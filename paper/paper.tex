%%
%% Beginning of file 'sample62.tex'
%%
%% Modified 2018 January
%%
%% This is a sample manuscript marked up using the
%% AASTeX v6.2 LaTeX 2e macros.
%%
%% AASTeX is now based on Alexey Vikhlinin's emulateapj.cls 
%% (Copyright 2000-2015).  See the classfile for details.

%% AASTeX requires revtex4-1.cls (http://publish.aps.org/revtex4/) and
%% other external packages (latexsym, graphicx, amssymb, longtable, and epsf).
%% All of these external packages should already be present in the modern TeX 
%% distributions.  If not they can also be obtained at www.ctan.org.

%% The first piece of markup in an AASTeX v6.x document is the \documentclass
%% command. LaTeX will ignore any data that comes before this command. The 
%% documentclass can take an optional argument to modify the output style.
%% The command below calls the preprint style  which will produce a tightly 
%% typeset, one-column, single-spaced document.  It is the default and thus
%% does not need to be explicitly stated.
%%
%%
%% using aastex version 6.2
\documentclass[modern]{aastex62}

\usepackage{url}
\usepackage{amsmath}
\usepackage{amssymb}
\usepackage{natbib}
\usepackage{multirow}
\bibliographystyle{aasjournal}


%% The default is a single spaced, 10 point font, single spaced article.
%% There are 5 other style options available via an optional argument. They
%% can be envoked like this:
%%
%% \documentclass[argument]{aastex62}
%% 
%% where the layout options are:
%%
%%  twocolumn   : two text columns, 10 point font, single spaced article.
%%                This is the most compact and represent the final published
%%                derived PDF copy of the accepted manuscript from the publisher
%%  manuscript  : one text column, 12 point font, double spaced article.
%%  preprint    : one text column, 12 point font, single spaced article.  
%%  preprint2   : two text columns, 12 point font, single spaced article.
%%  modern      : a stylish, single text column, 12 point font, article with
%% 		  wider left and right margins. This uses the Daniel
%% 		  Foreman-Mackey and David Hogg design.
%%  RNAAS       : Preferred style for Research Notes which are by design 
%%                lacking an abstract and brief. DO NOT use \begin{abstract}
%%                and \end{abstract} with this style.
%%
%% Note that you can submit to the AAS Journals in any of these 6 styles.
%%
%% There are other optional arguments one can envoke to allow other stylistic
%% actions. The available options are:
%%
%%  astrosymb    : Loads Astrosymb font and define \astrocommands. 
%%  tighten      : Makes baselineskip slightly smaller, only works with 
%%                 the twocolumn substyle.
%%  times        : uses times font instead of the default
%%  linenumbers  : turn on lineno package.
%%  trackchanges : required to see the revision mark up and print its output
%%  longauthor   : Do not use the more compressed footnote style (default) for 
%%                 the author/collaboration/affiliations. Instead print all
%%                 affiliation information after each name. Creates a much
%%                 long author list but may be desirable for short author papers
%%
%% these can be used in any combination, e.g.
%%
%% \documentclass[twocolumn,linenumbers,trackchanges]{aastex62}
%%
%% AASTeX v6.* now includes \hyperref support. While we have built in specific
%% defaults into the classfile you can manually override them with the
%% \hypersetup command. For example,
%%
%%\hypersetup{linkcolor=red,citecolor=green,filecolor=cyan,urlcolor=magenta}
%%
%% will change the color of the internal links to red, the links to the
%% bibliography to green, the file links to cyan, and the external links to
%% magenta. Additional information on \hyperref options can be found here:
%% https://www.tug.org/applications/hyperref/manual.html#x1-40003
%%
%% If you want to create your own macros, you can do so
%% using \newcommand. Your macros should appear before
%% the \begin{document} command.
%%

% todos
\newcommand{\todo}[3]{{\color{#2}\emph{#1}: #3}}
\newcommand{\agoltodo}[1]{\todo{Agol}{blue}{#1}}
\newcommand{\gordontodo}[1]{\todo{Gordon}{red}{#1}}

% projects
\newcommand{\project}[1]{\textsf{#1}}
\newcommand{\celerite}{\project{celerite }}

% useful commands
\newcommand{\bvec}[1]{{\ensuremath{\boldsymbol{#1}}}}
\newcommand{\T}{\ensuremath{\mathrm{T}}}
\newcommand{\expandvec}[2]{\left(\begin{array}{ccccc} #1\quad && \cdots\quad && #2 \end{array}\right)}

%% Tells LaTeX to search for image files in the 
%% current directory as well as in the figures/ folder.
\graphicspath{}

%% Reintroduced the \received and \accepted commands from AASTeX v5.2
\received{}
\revised{}
\accepted{}
%% Command to document which AAS Journal the manuscript was submitted to.
%% Adds "Submitted to " the arguement.
\submitjournal{}

%% Mark up commands to limit the number of authors on the front page.
%% Note that in AASTeX v6.2 a \collaboration call (see below) counts as
%% an author in this case.
%
%\AuthorCollaborationLimit=3
%
%% Will only show Schwarz, Muench and "the AAS Journals Data Scientist 
%% collaboration" on the front page of this example manuscript.
%%
%% Note that all of the author will be shown in the published article.
%% This feature is meant to be used prior to acceptance to make the
%% front end of a long author article more manageable. Please do not use
%% this functionality for manuscripts with less than 20 authors. Conversely,
%% please do use this when the number of authors exceeds 40.
%%
%% Use \allauthors at the manuscript end to show the full author list.
%% This command should only be used with \AuthorCollaborationLimit is used.

%% The following command can be used to set the latex table counters.  It
%% is needed in this document because it uses a mix of latex tabular and
%% AASTeX deluxetables.  In general it should not be needed.
%\setcounter{table}{1}

%%%%%%%%%%%%%%%%%%%%%%%%%%%%%%%%%%%%%%%%%%%%%%%%%%%%%%%%%%%%%%%%%%%%%%%%%%%%%%%%
%%
%% The following section outlines numerous optional output that
%% can be displayed in the front matter or as running meta-data.
%%
%% If you wish, you may supply running head information, although
%% this information may be modified by the editorial offices.
\shorttitle{Celerite2D}
\shortauthors{Gordon et al.}
%%
%% You can add a light gray and diagonal water-mark to the first page 
%% with this command:
% \watermark{text}
%% where "text", e.g. DRAFT, is the text to appear.  If the text is 
%% long you can control the water-mark size with:
%  \setwatermarkfontsize{dimension}
%% where dimension is any recognized LaTeX dimension, e.g. pt, in, etc.
%%
%%%%%%%%%%%%%%%%%%%%%%%%%%%%%%%%%%%%%%%%%%%%%%%%%%%%%%%%%%%%%%%%%%%%%%%%%%%%%%%%

%% This is the end of the preamble.  Indicate the beginning of the
%% manuscript itself with \begin{document}.

\begin{document}

\title{}

%% LaTeX will automatically break titles if they run longer than
%% one line. However, you may use \\ to force a line break if
%% you desire. In v6.2 you can include a footnote in the title.

%% A significant change from earlier AASTEX versions is in the structure for 
%% calling author and affilations. The change was necessary to implement 
%% autoindexing of affilations which prior was a manual process that could 
%% easily be tedious in large author manuscripts.
%%
%% The \author command is the same as before except it now takes an optional
%% arguement which is the 16 digit ORCID. The syntax is:
%% \author[xxxx-xxxx-xxxx-xxxx]{Author Name}
%%
%% This will hyperlink the author name to the author's ORCID page. Note that
%% during compilation, LaTeX will do some limited checking of the format of
%% the ID to make sure it is valid.
%%
%% Use \affiliation for affiliation information. The old \affil is now aliased
%% to \affiliation. AASTeX v6.2 will automatically index these in the header.
%% When a duplicate is found its index will be the same as its previous entry.
%%
%% Note that \altaffilmark and \altaffiltext have been removed and thus 
%% can not be used to document secondary affiliations. If they are used latex
%% will issue a specific error message and quit. Please use multiple 
%% \affiliation calls for to document more than one affiliation.
%%
%% The new \altaffiliation can be used to indicate some secondary information
%% such as fellowships. This command produces a non-numeric footnote that is
%% set away from the numeric \affiliation footnotes.  NOTE that if an
%% \altaffiliation command is used it must come BEFORE the \affiliation call,
%% right after the \author command, in order to place the footnotes in
%% the proper location.
%%
%% Use \email to set provide email addresses. Each \email will appear on its
%% own line so you can put multiple email address in one \email call. A new
%% \correspondingauthor command is available in V6.2 to identify the
%% corresponding author of the manuscript. It is the author's responsibility
%% to make sure this name is also in the author list.
%%
%% While authors can be grouped inside the same \author and \affiliation
%% commands it is better to have a single author for each. This allows for
%% one to exploit all the new benefits and should make book-keeping easier.
%%
%% If done correctly the peer review system will be able to
%% automatically put the author and affiliation information from the manuscript
%% and save the corresponding author the trouble of entering it by hand.

\correspondingauthor{Tyler A. Gordon}
\email{tagordon@uw.edu}

\author{Tyler A. Gordon}
\affiliation{Department of Astronomy, University of Washington, Box 351580, U.W., Seattle, WA 98195-1580, USA}

\author{Eric Agol}
\affiliation{Department of Astronomy, University of Washington, Box 351580, U.W., Seattle, WA 98195-1580, USA}

%% Note that the \and command from previous versions of AASTeX is now
%% depreciated in this version as it is no longer necessary. AASTeX 
%% automatically takes care of all commas and "and"s between authors names.

%% AASTeX 6.2 has the new \collaboration and \nocollaboration commands to
%% provide the collaboration status of a group of authors. These commands 
%% can be used either before or after the list of corresponding authors. The
%% argument for \collaboration is the collaboration identifier. Authors are
%% encouraged to surround collaboration identifiers with ()s. The 
%% \nocollaboration command takes no argument and exists to indicate that
%% the nearby authors are not part of surrounding collaborations.

%% Mark off the abstract in the ``abstract'' environment. 
\begin{abstract}
Gaussian processes are frequently used to model variability in astrophysical time-series. In addition to studying variability in stars and AGN, Gaussian 
process regression can be used to account for correlated noise in light curves of exoplanetary transits and micro-lensing events. One of the chief limitations of this 
technique is the computational time required to compute a Gaussian process model for large data-sets. This also limits prospects for using Gaussian processes with 
multi-dimensional datasets, although sparse-matrix approximations can make GP regression on multi-dimensional datasets feasible for some applications. Here we 
present a method based on the \celerite GP implementation which enables fast computation of two-dimensional gaussian process noise models for datasets with a 
small second dimension, such as multi-band photometry. This method enables the use of GP regression with multi-wavelength transit and micro-lensing light curves, 
and may have applications to radial velocity and direct imaging data as well. 

\end{abstract}

%% Keywords should appear after the \end{abstract} command. 
%% See the online documentation for the full list of available subject
%% keywords and the rules for their use.
%\keywords{}

%% From the front matter, we move on to the body of the paper.
%% Sections are demarcated by \section and \subsection, respectively.
%% Observe the use of the LaTeX \label
%% command after the \subsection to give a symbolic KEY to the
%% subsection for cross-referencing in a \ref command.
%% You can use LaTeX's \ref and \label commands to keep track of
%% cross-references to sections, equations, tables, and figures.
%% That way, if you change the order of any elements, LaTeX will
%% automatically renumber them.
%%
%% We recommend that authors also use the natbib \citep
%% and \citet commands to identify citations.  The citations are
%% tied to the reference list via symbolic KEYs. The KEY corresponds
%% to the KEY in the \bibitem in the reference list below. 
\tableofcontents

\section{introduction}\label{sec:intro}
	A Gaussian process is a generalization of the Gaussian distribution to functions \citep{Rasmussen:2006}. They have been used 
	extensively in astrophysics as a model of stochastic variability across both space and time to study variable stars, AGN, the extragalactic 
	dust distribution, and to model correlated noise in photometric and radial velocity measurements of exoplanet hosts. \gordontodo{X-ray binaries?}
	
	\gordontodo{something about how a GP model is non-parametric and the advantages thereof? Also agnostic 
	to the source of variability (seems related) --- essentially: what makes a GP noise model desirable over alternatives?}
	
	Unfortunately, most GP operations are computationally expensive. The common tasks of computing a likelihood function, sampling from a GP, and 
	interpolating or smoothing data all run in $\mathcal{O}(N^3)$ time for $N$ data points \gordontodo{make sure this is true for interpolation}. This is 
	especially problematic for use cases that require iterative calls to the likelihood function, as is the case when minimizing the likelihood or performing 
	Markov Chain Monte Carlo simulations. This has prevented the adoption of GP methods for datasets larger than $N \sim 10^3$. 
	
	The issue of computational expense can be partially overcome either by approximating a GP (often by exploiting the properties of sparse 
	matrices), or by restricting the user to a subset of GPs for which the computation can be sped up. One member of the latter class is the 
	\celerite method, which increases the speed of most common GP computations to $\mathcal{O}(NJ^2)$ where $J$ is typically small compared 
	to $N$. \celerite achieves this increased speed by limiting the user to GPs with a specific functional form of the covariance, albeit a useful and 
	versatile one. Additionally, the \celerite method is only applicable to one-dimensional GPs. Here we present a method to compute a two-dimensional 
	GP on a grid in $\mathcal{O}(NM^3J^2)$ time for an $N\times M$ grid when the covariance in one of the dimensions is described by the \celerite 
	kernel. This is an improvement over the general case in which the computation runs in $\mathcal{O}(N^3M^3)$. Our method is especially useful 
	in cases where $M << N$ and $M$ is constant. In this case the factor of $M^3$ in the runtime can be thought of as a constant. An example would 
	be modeling variability in multiband photometry where $M$ would represent the number of bands and would be small compared to $N$. 
		
	\gordontodo{This is where we come in...}
		
\section{gaussian processes and the celerite model}
	A Gaussian process is a stochastic noise model which consists of a mean:
	\begin{equation} 
		\bvec{\mu_\theta}(\bvec{x}) = \expandvec{\mu_\bvec{\theta}(x_1)}{\mu_\bvec{\theta}(x_N)}
	\end{equation}
	and covariance defined the kernel function $k_\bvec{\alpha}(x_n, x_m)$. The mean and kernel functions depend on the parameters 
	$\bvec{\theta}$ and $\bvec{\alpha}$ respectively, which are referred to as ``hyperparameters" of the GP. The kernel function defines a covariance matrix 
	$\left[K\right]_{n, m} = k_\alpha(x_n, x_m)$. This model has a likelihood function given by
	\begin{equation}
		\ln\ \mathcal{L(\bvec{\theta}, \bvec{\alpha})} = \ln\ p(\bvec{y}|X, \bvec{\theta}, \bvec{\alpha}) = 
			-\frac{1}{2}\bvec{r_\theta}^\T K_\bvec{\alpha}^{-1}\bvec{r_\theta} 
			-\frac{1}{2}\ln\ \mathrm{det}(K_\bvec{\alpha}) - \frac{N}{2}\ln(2\pi)
	\end{equation}
	\gordontodo{I'm using $X$ here because DFM does, but wouldn't $\bvec{x}$ be more clear?}
	where 
	\begin{equation}
		\bvec{y} = \expandvec{y_1}{y_N}
	\end{equation}
	are the data, $N$ is the number of datapoints, and
	\begin{equation}
		\bvec{r_\theta} = \bvec{y}-\bvec{\mu_\theta}(\bvec{x})
	\end{equation}
	By maximizing the likelihood (or, in practice, minimizing the negative of the log-likelihood), one can obtain an estimate of the hyperparameters $\bvec{\theta}$ and 
	$\bvec{\alpha}$. Furthermore, Markov Chain Monte Carlo (MCMC) methods can be used to sample the likelihood in hyperparameter-space. \gordontodo{expand 
	on this.}
	\subsection{The \celerite model}
		The primary limitation of Gaussian process noise models is that they are computationally expensive, with computation of the likelihood function scaling as 
		$\mathcal{O}(N^3)$ due to the matrix inversion. This is especially problematic for use cases such as MCMC which require calling the likelihood function many 
		times. This has prevented the adoption of Gaussian process noise models for datasets larger than $N \approx 10^3$, except in cases where approximate 
		methods suffice. 
		
		The \celerite method overcomes this issue by reducing the time to compute the likelihood function to $\mathcal{O}(NJ^2)$ for one 
		class of kernel functions; where J is the number of \celerite terms making up the kernel function of the GP. This class of kernel functions are given by 
		\begin{equation}
			k_\alpha(t_n, t_m) = \sigma_n^2 \delta_{nm} + \sum_{j=1}^J a_j e^{-c_j\tau_{nm}}
		\end{equation}
		where $\bvec{\alpha} = (a_1...a_j, c_1...c_j)$, $\sigma_n^2$ is the variance of the white noise, and $\tau_{nm} = |t_n-t_m|$. 
		The coefficients $a$ and $c$ may be complex.
	
		For kernel functions of this form, the covariance matrix can be written as a semi-separable matrix of rank $2J$:
		\begin{equation}
			K = A + \mathrm{tril}(UV^\T) + \mathrm{triu}(VU^\T)
		\end{equation}
		where $U$ and $V$ have entries given by: 
		\begin{eqnarray}
			U_{n, 2j-1} &=& a_je^{-c_jt_n}\cos(d_jt_n) + b_je^{-c_jt_n}\sin(d_jt_n) \nonumber \\
			U_{n, 2j} &=& a_je^{-c_jt_n}\sin(d_jt_n) - b_je^{-c_jt_n}\cos(d_jt_n) \nonumber \\
			V_{m, 2j-1} &=& e^{c_jt_m}\cos(d_jt_m) \nonumber \\
			V_{m, 2j} &=& e^{c_jt_m}\sin(d_jt_m)
		\end{eqnarray}
		and A is a diagonal matrix:
		\begin{equation}
			A_{n,n} = \sigma_n^2 + \sum_{j=1}^Ja_j
		\end{equation}
		Cholesky decomposition can be performed on this matrix in $\mathcal{O}(NJ^2)$ time, as described in the appendix. 
		 
		\gordontodo{Add discussion of the versatility of this kernel, and it's connection to stochastically driven harmonic oscillators.}
		
		\gordontodo{Add discussion of semiseperability --- the reader should understand that the limitation of \celerite is that only kernels of this form can be 
		represented by a semi-seperable matrix, which is what makes moving to two dimensions slightly more complicated than ordinary.}
		

\section{2d gaussian processes}
	\subsection{kronecker structure of the 2d covariance matrix}
	In two dimensions the kernel function depends on two coordinates. We will continue to use
	\begin{equation}
		\bvec{t} = \expandvec{t_1}{t_N}
	\end{equation}
	to refer to the first coordinate, and introduce
	\begin{equation}
		\bvec{u} = \expandvec{u_1}{u_M}
	\end{equation}
	to refer to the second coordinate. In most cases, we will demand that $M << N$. We then define $\bvec{x}$ to be the list of points
	\begin{equation}
		\bvec{x} = \expandvec{(t_1, u_1), (t_1, u_2), \cdots (t_1, u_M)}{(t_N, u_M)}
	\end{equation}
	We can also write this as
	\begin{equation}
		x_i = (t_{\lfloor(i-1)/M\rfloor + 1}, u_{i - \lfloor(i-1)/M\rfloor M})
	\end{equation} 
	The covariance between $x_i$ and $x_j$ is then given by
	\begin{equation}
		k_{\bvec{\alpha}\bvec{\beta}}(x_i, x_j) = \sigma_i^2\delta_{ij} + q_\bvec{\alpha}(\tau_{ij})\circ r_\bvec{\beta}(\nu_{ij})
	\end{equation}
	where $\tau_{ij} = |t_{\lfloor(i-1)/M\rfloor + 1}-t_{\lfloor(j-1)/M\rfloor + 1}|$, 
	$\nu_{ij} = |u_{i - \lfloor(i-1)/M\rfloor M}-u_{j - \lfloor(j-1)/M\rfloor M}|$; 
	$q$ and $r$ are one-dimensional kernel functions which depend on the hyperparameters $\bvec{\alpha}$ and 
	$\bvec{\beta}$ respectively; 
	and $\circ$ is some binary operation. 
	If $Q_\bvec{\alpha}$ and $R_\bvec{\alpha}$ are the covariance matrices corresponding to 
	$q_\bvec{\alpha}$ and $r_\bvec{\alpha}$ such that
	\begin{equation}
		\left[Q_\bvec{\alpha}\right]_{ij} = q_\bvec{\alpha}(\tau_{ij})\ \ \ \mathrm{and}\ \ \ \left[R_\bvec{\beta}\right]_{ij} = r_\bvec{\beta}(\tau_{ij})
	\end{equation} 
	then, using $q_{ij}$ and $r_{ij}$ as shorthand for the entries in $Q_\bvec{\alpha}$ and $R_\bvec{\beta}$, $K_\bvec{\alpha\beta}$ can be represented as: 
	\begin{equation}
		K_\bvec{\alpha\beta} = \Sigma + 
			\begin{bmatrix}
				q_{11}\circ R_\bvec{\beta} & ... & q_{1N}\circ R_\bvec{\beta} \\
				\vdots & \ddots & \\
				q_{N1}\circ R_\bvec{\beta} & ... & q_{NN}\circ R_\bvec{\beta}
			\end{bmatrix}
	\end{equation}
	where $[\Sigma]_{ij} = \sigma_i^2\delta_{ij}$ and
	\begin{equation}
		q_{pq}\circ R_\bvec{\alpha} = 
			\begin{bmatrix}
				q_{pq}\circ r_{11} & ... & q_{pq}\circ r_{1M} \\
				\vdots & \ddots & \\
				q_{pq}\circ r_{M1} & ... & q_{pq}\circ r_{MM}
			\end{bmatrix}
	\end{equation}
	If the operator $\circ$ represents multiplication, we have 
	\begin{equation}
		K_\bvec{\alpha\beta} = \Sigma + Q_\bvec{\alpha} \otimes  R_\bvec{\beta}
	\end{equation}
	where $\otimes$ is the Kronecker product. If the covariance in $\bvec{t}$ is described by a \celerite kernel can write 
	\begin{eqnarray}
		K &=& \Sigma + \left[\mathrm{tril}(UV^\T) + \mathrm{triu}(VU^\T)\right]\otimes R \\
		&=& \Sigma + \mathrm{tril}\left[ (U\otimes R)(V\otimes I_M)^\T\right] + \mathrm{triu}\left[ (V\otimes I_M)(U\otimes R)^\T\right]
	\end{eqnarray}
	This covariance has the same semi-separable form as in the one-dimensional case, but now $K$ is an $(NM\times NM)$ matrix of rank $2JM$. 
	This means that we can apply the same Cholesky decomposition algorithm (described in the appendix) as in the 1d case to compute the GP in 
	$\mathcal{O}(NJ^2M^3)$. While our implementation only handles the Kronecker product case, 
	his result holds in the case that $\circ$ is an operation other than multiplication, so long as $\circ$ has the property
	\begin{equation}
		(a\circ B)(c\circ D) = (ac)\circ(BD)
	\end{equation}
	for scalars $a$ and $b$ and matrices $B$ and $D$. 
	%where $a\circ B$ is the matrix 
	%\begin{equation}
	%	a\circ B = \begin{bmatrix}
	%		a\circ b_{11} & a\circ b_{12} & \cdots \\
	%		a\circ b_{21} &  a\circ b_{22} & \\
	%		\vdots & & \ddots & 
	%	\end{bmatrix}
	%\end{equation}
	\gordontodo{See appendix for proof}
	
	\subsection{sampling the distribution}
	A gaussian process can be used to simulate correlated noise. If $L$ is the lower triangular factor from the Cholesky decomposition of the 
	$(N\times N)$ covariance matrix $K$, and $\bvec{z}$ is a vector of length $N$ containing random numbers drawn for a standard normal distribution, i.e.
	\begin{equation}
		\bvec{z} \sim \mathcal{N}(1, 0)
	\end{equation}
	then $\bvec{x} = \bvec{\mu_\bvec{\theta}(\bvec{x})} + L\bvec{z}$ is a vector with entries correlated by $K$ and with mean $\bvec{\mu_\theta}$, i.e.
	\begin{equation}
		\bvec{x} \sim \mathcal{N}(\bvec{\mu_\theta}, K)
	\end{equation}
	\gordontodo{fix this so that $\mu$ depends on $\theta$ and is a function of $\bvec{t}$ or something like that. Compuational cost should be $\mathcal{O}(NJM^2)$}
	
	\subsection{interpolation and extrapolation}	
	\gordontodo{Bit more complicated --- if both dimensions are celerite processes then the scaling is better than in the case where $R$ is not celerite. You have notes on this.}
	
\section{example: exomoons}
	\gordontodo{What if I used Kepler-1625 as an example? Could simulate WFC3 GRISM data and show that moon could be validated with Celerite2D?}
	Gaussian process noise models are most advantageous when correlated noise dominates over white noise. 
	This is already the case for state-of-the-art radial velocity observations, and transit 
	photometry is not far behind. Future space-based observatories such as James Webb 
	will enable high-precision photometry in which stellar variability is the main barrier to detecting ever more shallow transit signals. 
	Stellar variability which obscures transit signal can be a result of spot and faculae crossings, asteroseismic oscillations, or variable 
	granulation on the star's surface. 
	
	\gordontodo{Are GPs applicable to instrumental noise as well? Seems best to account for instrumental effects with models that account for known factors since 
	they might not be well-described by a gaussian process, right?}
	
	\gordontodo{What about other methods of accounting for astrophysical noise? i.e. cosine filtering, polynomials, etc. --- also related to GP as non-parametric 
	noise model}

\section{discussion}
	\subsection{other applications}

\section{conclusion}

%% If you wish to include an acknowledgments section in your paper,
%% separate it off from the body of the text using the \acknowledgments
%% command.
\acknowledgments

%% To help institutions obtain information on the effectiveness of their 
%% telescopes the AAS Journals has created a group of keywords for telescope 
%% facilities.
%
%% Following the acknowledgments section, use the following syntax and the
%% \facility{} or \facilities{} macros to list the keywords of facilities used 
%% in the research for the paper.  Each keyword is check against the master 
%% list during copy editing.  Individual instruments can be provided in 
%% parentheses, after the keyword, but they are not verified.

%\vspace{5mm}
%\facilities{}

%% Similar to \facility{}, there is the optional \software command to allow 
%% authors a place to specify which programs were used during the creation of 
%% the manusscript. Authors should list each code and include either a
%% citation or url to the code inside ()s when available.

%\software{}

%% Appendix material should be preceded with a single \appendix command.
%% There should be a \section command for each appendix. Mark appendix
%% subsections with the same markup you use in the main body of the paper.

%% Each Appendix (indicated with \section) will be lettered A, B, C, etc.
%% The equation counter will reset when it encounters the \appendix
%% command and will number appendix equations (A1), (A2), etc. The
%% Figure and Table counter will not reset.

\appendix

%% The reference list follows the main body and any appendices.
%% Use LaTeX's thebibliography environment to mark up your reference list.
%% Note \begin{thebibliography} is followed by an empty set of
%% curly braces.  If you forget this, LaTeX will generate the error
%% "Perhaps a missing \item?".
%%
%% thebibliography produces citations in the text using \bibitem-\cite
%% cross-referencing. Each reference is preceded by a
%% \bibitem command that defines in curly braces the KEY that corresponds
%% to the KEY in the \cite commands (see the first section above).
%% Make sure that you provide a unique KEY for every \bibitem or else the
%% paper will not LaTeX. The square brackets should contain
%% the citation text that LaTeX will insert in
%% place of the \cite commands.

%% We have used macros to produce journal name abbreviations.
%% \aastex provides a number of these for the more frequently-cited journals.
%% See the Author Guide for a list of them.

%% Note that the style of the \bibitem labels (in []) is slightly
%% different from previous examples.  The natbib system solves a host
%% of citation expression problems, but it is necessary to clearly
%% delimit the year from the author name used in the citation.
%% See the natbib documentation for more details and options.

\begin{thebibliography}{}

% \bibitem[Astropy Collaboration et al.(2013)]{2013A&A...558A..33A} Astropy Collaboration, Robitaille, T.~P., Tollerud, E.~J., et al.\ 2013, \aap, 558, A33 

\end{thebibliography}

%% This command is needed to show the entire author+affilation list when
%% the collaboration and author truncation commands are used.  It has to
%% go at the end of the manuscript.
%\allauthors

%% Include this line if you are using the \added, \replaced, \deleted
%% commands to see a summary list of all changes at the end of the article.
%\listofchanges

\end{document}

% End of file `sample62.tex'.
